\section{Примеры создания таблиц} \label{app:tables}
\sectionmark{Примеры создания таблиц}

В этом разделе приводятся таблицы, разной сложности.

\subsection{Таблицы общего назначения}

Таблица~\ref{appendix:t:1} простая.

\begin{longtable}{|p{60mm}|p{100mm}|}
  \caption{Простая} \label{appendix:t:1} \\
  \hline
  \multicolumn{1}{|p{60mm}|}{\centering Обозначение} &
  \multicolumn{1}{p{100mm}|}{\centering Наименование} \\\hline
  \endfirsthead
  \caption*{Продолжение таблицы \ref{appendix:t:1}} \\
 \hline
  \multicolumn{1}{|p{60mm}|}{\centering Обозначение} &
  \multicolumn{1}{p{100mm}|}{\centering Наименование} \\\hline
  \endhead
   1    &   2   \\ \hline
   3    &   4   \\ \hline
   5    &   6   \\ \hline
\end{longtable}

Таблица~\ref{appendix:t:2} более сложная.

\begin{longtable}{|p{50mm}|p{25mm}|p{50mm}|p{25mm}|}
  \caption{Посложнее} \label{appendix:t:2} \\
  \hline
  \multicolumn{1}{|p{50mm}|}{Наименование файла} &
  \multicolumn{1}{p{25mm}|}{\centering Объем, байт} & 
  \multicolumn{1}{p{50mm}|}{\centering Контрольная сумма по программе CRC32} &
  \multicolumn{1}{p{25mm}|}{Примечание} \\\hline
  \endfirsthead
  \caption*{Продолжение таблицы \ref{appendix:t:2}} \\
 \hline
  \multicolumn{1}{|p{50mm}|}{Наименование файла} &
  \multicolumn{1}{p{25mm}|}{\centering Объем, байт} & 
  \multicolumn{1}{p{50mm}|}{\centering Контрольная сумма по программе CRC32} &
  \multicolumn{1}{p{25mm}|}{Примечание} \\\hline
  \endhead
   1   &  \centering 2   &  3    &  4      \\ \hline
   5   &  \centering 6   &  7    &  8      \\ \hline
   9   &  \centering 10  &  11   &  12     \\ \hline
\end{longtable}


Таблица~\ref{appendix:t:3} ещё сложнее.

\newcolumntype{L}[1]{>{\raggedright\arraybackslash}p{#1}}% Выравнивание столбца по левому краю
\newcolumntype{C}[1]{>{\centering\arraybackslash}p{#1}}% Выравнивание столбца по левому центру
\newcolumntype{R}[1]{>{\raggedleft\arraybackslash}p{#1}}% Выравнивание столбца по правому краю


\begin{longtable}{|R{75mm}|C{25mm}|L{25mm}|}
  \caption{Сложноватая} \label{appendix:t:31} \\
  \hline
  & \multicolumn{2}{p{50mm}|}{\centering Номер пункта инструкции} 
              \\ \cline{2-3}
  \multicolumn{1}{|p{75mm}|}{Наименование испытаний и проверок} &
  \multicolumn{1}{p{25mm}|}{\centering раздела ''Требования к изделиям''} & 
  \multicolumn{1}{p{25mm}|}{\centering раздела ''Методы испытаний''} \\\hline
  \endfirsthead
  \caption*{Продолжение таблицы \ref{appendix:t:3}} \\
 \hline
  &
  \multicolumn{2}{p{50mm}|}{\centering Номер пункта инструкции} 
              \\\cline{2-3}
  \multicolumn{1}{|p{75mm}|}{Наименование испытаний и проверок} &
  \multicolumn{1}{p{25mm}|}{\centering раздела ''Требования к изделиям''} & 
  \multicolumn{1}{p{25mm}|}{\centering раздела ''Методы испытаний''} \\\hline
  \endhead
  \multicolumn{3}{|p{145mm}|}
  {%\parbox{170cm}
  {Примечание – Последовательность испытаний может изменяться по согласованию с кем – нибудь.}} \\ \hline
\endlastfoot
   1  &  4  &  7    \\ \hline
   2  &  5  &  8    \\ \hline
   3  &  6  &  9    \\ \hline
\end{longtable}



\begin{longtable}{|R{75mm}|C{25mm}|L{25mm}|p{25mm}|}
  \caption{Сложноватая} \label{appendix:t:3} \\
  \hline
  &
  \multicolumn{2}{p{50mm}|}{\centering Номер пункта инструкции} &
              \\\cline{2-3}
  \multicolumn{1}{|p{75mm}|}{Наименование испытаний и проверок} &
  \multicolumn{1}{p{25mm}|}{\centering раздела ''Требования к изделиям''} & 
  \multicolumn{1}{p{25mm}|}{\centering раздела ''Методы испытаний''} &
  \multicolumn{1}{p{25mm}|}{\centering Примечание} \\\hline
  \endfirsthead
  \caption*{Продолжение таблицы \ref{appendix:t:3}} \\
 \hline
  &
  \multicolumn{2}{p{50mm}|}{\centering Номер пункта инструкции} &
              \\\cline{2-3}
  \multicolumn{1}{|p{75mm}|}{Наименование испытаний и проверок} &
  \multicolumn{1}{p{25mm}|}{\centering раздела ''Требования к изделиям''} & 
  \multicolumn{1}{p{25mm}|}{\centering раздела ''Методы испытаний''} &
  \multicolumn{1}{p{25mm}|}{\centering Примечание} \\\hline
  \endhead
  \multicolumn{4}{|p{170mm}|}
  {%\parbox{170cm}
  {Примечание – Последовательность испытаний может изменяться по согласованию с кем – нибудь.}} \\ \hline
\endlastfoot
   1  &  4  &  7  &  10  \\ \hline
   2  &  5  &  8  &  11  \\ \hline
   3  &  6  &  9  &  12  \\ \hline
\end{longtable}


Таблица~\ref{t:haracteristics} одна из самых сложных в приложении~\ref{app:tables}.

\begin{longtable}{|p{42mm}|p{25mm}|p{25mm}|p{25mm}|p{10mm}|p{10mm}|p{10mm}|}
  \caption{Совсем непростая} \label{t:haracteristics} \\
  \hline
  &  \multicolumn{2}{p{50mm}|}{Основные характеристики} &  &  &   &       \\ \cline{2-3}
 
  \multicolumn{1}{|p{42mm}|}{\parbox[b]{4.0cm}{Наименование и обозначение средства
                             измерения, контроля, испытания,
                             вспомогательного оборудования}} &
  \multicolumn{1}{p{25mm}|}{\centering \rotatebox{90}{\parbox{4cm}{Класс точности}}} & 
  \multicolumn{1}{p{25mm}|}{\centering \rotatebox{90}{\parbox{4cm}{Используемые \\ характеристики}}} &
  \multicolumn{1}{p{25mm}|}{\rotatebox{90}{\parbox{4cm}{Тип, соответствующий требованиям и основным характеристикам}}} &
  \multicolumn{1}{p{10mm}|}{\rotatebox{90}{\parbox{4cm}{Номер пункта  \\ методов}}} &
  \multicolumn{1}{p{10mm}|}{\rotatebox{90}{\parbox{4cm}{Количество на одно рабочее место}}} &
  \multicolumn{1}{p{10mm}|}{\rotatebox{90}{\parbox{4cm}{Примечание}}} \\ \hline
  \endfirsthead

  \caption*{Продолжение таблицы \ref{t:haracteristics}} \\
  \hline
  &  \multicolumn{2}{p{50mm}|}{Основные характеристики} &  &  &   &       \\ \cline{2-3}
 
  \multicolumn{1}{|p{42mm}|}{\parbox[b]{4.0cm}{Наименование и обозначение средства
                             измерения, контроля, испытания,
                             вспомогательного оборудования}} &
  \multicolumn{1}{p{25mm}|}{\centering \rotatebox{90}{\parbox{4cm}{Класс точности}}} & 
  \multicolumn{1}{p{25mm}|}{\centering \rotatebox{90}{\parbox{4cm}{Используемые \\ характеристики}}} &
  \multicolumn{1}{p{25mm}|}{\rotatebox{90}{\parbox{4cm}{Тип, соответствующий требованиям и основным характеристикам}}} &
  \multicolumn{1}{p{10mm}|}{\rotatebox{90}{\parbox{4cm}{Номер пункта  \\ методов}}} &
  \multicolumn{1}{p{10mm}|}{\rotatebox{90}{\parbox{4cm}{Количество на одно рабочее место}}} &
  \multicolumn{1}{p{10mm}|}{\rotatebox{90}{\parbox{4cm}{Примечание}}} \\ \hline
  \endhead
  
  \multicolumn{7}{|p{170mm}|}
  {%\parbox{170cm}
  {Примечание – Средства измерений, контроля и испытаний, а также вспомогательное оборудование в 
  процессе работы могут заменяться средствами другого типа, обеспечивающими
  необходимую погрешность измерения и удовлетворяющими условиям испытаний.}} \\ \hline
\endlastfoot
       &    &    &    &   &   & \\ \hline
       &    &    &    &   &   & \\ \hline
       &    &    &    &   &   & \\ \hline
       &    &    &    &   &   & \\ \hline
\end{longtable}




В таблице~\ref{t:par1} заголовочные надписи лучше выровнены.

\begin{longtable}{|p{45mm}|p{25mm}|p{25mm}|p{25mm}|p{30mm}|}
  \caption{} \label{t:par1} \\
  \hline
  \multicolumn{1}{|p{45mm}|}{\raisebox{-5mm}[0mm][6mm]{\parbox{45mm}{\small Наименование параметра, единица измерения}}} & \multicolumn{2}{p{55mm}|}{\small Величина параметра}                                                      &   \multicolumn{1}{p{25mm}|}{\raisebox{-8mm}[0mm][0mm]{\parbox{25mm}{\small Допустимая погрешность измерения}}} &   \multicolumn{1}{p{25mm}|}{\small Примечание} \\
  \cline{2-3} 
                                                                        & \multicolumn{1}{p{25mm}|}{\small Номинальное значение} & \multicolumn{1}{p{25mm}|}{\small Предельное отклонение} &                                                               &                                         \\\hline
  \endfirsthead

  \caption*{Продолжение таблицы \ref{t:par1}} \\
  \hline
  \multicolumn{1}{|p{45mm}|}{\raisebox{-5mm}[0mm][6mm]{\parbox{45mm}{\small Наименование параметра, единица измерения}}} & \multicolumn{2}{p{55mm}|}{\small Величина параметра}                                                      &   \multicolumn{1}{p{25mm}|}{\raisebox{-8mm}[0mm][0mm]{\parbox{25mm}{\small Допустимая погрешность измерения}}} &   \multicolumn{1}{p{25mm}|}{\small Примечание} \\
  \cline{2-3} 
                                                                        & \multicolumn{1}{p{25mm}|}{\small Номинальное значение} & \multicolumn{1}{p{25mm}|}{\small Предельное отклонение} &                                                               &                                         \\\hline
  \endhead  
     Ток потребления номинальный, мА											          & 750                   &       \textpm50       &                                  &     \\\hline
     Ток потребления максимальный, мА											          & 950                   &       \textpm100       &                                  &    \\\hline
\end{longtable}




В таблице~\ref{t:har2} заголовочные надписи повёрнуты и выровнены.


\begin{longtable}{|p{42mm}|p{22mm}|p{22mm}|p{25mm}|p{16mm}|p{10mm}|p{10mm}|}
  \caption{} \label{t:har2} \\
  \hline
  &  \multicolumn{2}{p{44mm}|}{Основные характеристики} &  &  &   &       \\ \cline{2-3}
 
  \multicolumn{1}{|p{42mm}|}{\raisebox{4mm}[0mm][0mm]{\parbox[b]{4.0cm}{\small Наименование и обозначение средства
                             измерения, контроля, испытания,
                             вспомогательного оборудования}}} &
  \multicolumn{1}{p{22mm}|}{\rotatebox{90}{\parbox{3cm}{\small Класс \\точности}}} & 
  \multicolumn{1}{p{22mm}|}{\rotatebox{90}{\parbox{3cm}{\footnotesize Используемые \\ характеристики}}} &
  \multicolumn{1}{p{25mm}|}{\raisebox{0mm}[0mm][0mm]{\rotatebox{90}{\parbox{4cm}{\small Тип, соответствующий требованиям и основным характеристикам}}}} &
  \multicolumn{1}{p{16mm}|}{\raisebox{0mm}[0mm][0mm]{\rotatebox{90}{\parbox{4cm}{\small Номер пункта  \\ методов}}}} &
  \multicolumn{1}{p{10mm}|}{\raisebox{0mm}[0mm][0mm]{\rotatebox{90}{\parbox{4cm}{\small Количество на одно рабочее место}}}} &
  \multicolumn{1}{p{10mm}|}{\raisebox{0mm}[0mm][0mm]{\rotatebox{90}{\parbox{4cm}{\small Примечание}}}} \\ \hline
  \endfirsthead

  \caption*{Продолжение таблицы \ref{t:har2}} \\
  \hline
  &  \multicolumn{2}{p{44mm}|}{Основные характеристики} &  &  &   &       \\ \cline{2-3}
 
  \multicolumn{1}{|p{42mm}|}{\raisebox{4mm}[0mm][0mm]{\parbox[b]{4.0cm}{\small Наименование и обозначение средства
                             измерения, контроля, испытания,
                             вспомогательного оборудования}}} &
  \multicolumn{1}{p{22mm}|}{\rotatebox{90}{\parbox{3cm}{\small Класс \\точности}}} & 
  \multicolumn{1}{p{22mm}|}{\rotatebox{90}{\parbox{3cm}{\footnotesize Используемые \\ характеристики}}} &
  \multicolumn{1}{p{25mm}|}{\raisebox{0mm}[0mm][0mm]{\rotatebox{90}{\parbox{4cm}{\small Тип, соответствующий требованиям и основным характеристикам}}}} &
  \multicolumn{1}{p{16mm}|}{\raisebox{0mm}[0mm][0mm]{\rotatebox{90}{\parbox{4cm}{\small Номер пункта  \\ методов}}}} &
  \multicolumn{1}{p{10mm}|}{\raisebox{0mm}[0mm][0mm]{\rotatebox{90}{\parbox{4cm}{\small Количество на одно рабочее место}}}} &
  \multicolumn{1}{p{10mm}|}{\raisebox{0mm}[0mm][0mm]{\rotatebox{90}{\parbox{4cm}{\small Примечание}}}} \\ \hline
  \endhead
  
  \multicolumn{7}{|p{170mm}|}
  {%\parbox{170cm}
  {\hspace*{10mm} Примечание – Средства измерений, контроля и испытаний, а также вспомогательное оборудование в процессе работы могут заменяться в установленном порядке средствами другого типа, обеспечивающими необходимую погрешность измерения и удовлетворяющими условиям испытаний. }} \\ \hline
\endlastfoot

Источник питания постоянного тока 
 
                                   &                    & напряжение 7,5~В, ток до 1,5~А                                 &         &  & 1 &  \\ \hline

\end{longtable}



\newpage
\subsection{Специализированные таблицы}


\definecolor{ColorModbusSlaveAddr}{rgb}{0.588235, 0.788235, 0.690196}%Подобрать цвет поможет GIMP
\definecolor{ColorModbusFunct}{rgb}{0.647059, 0.819608, 0.870588}%Подобрать цвет поможет GIMP
\definecolor{ColorModbusData}{rgb}{1.000000, 0.752941, 0.796078}%Подобрать цвет поможет GIMP
\definecolor{ColorModbusChecksumm}{rgb}{0.968627, 0.980392, 0.725490}%Подобрать цвет поможет GIMP



\renewcommand\tablename{\hspace{0mm} Таблица}
\setlength\LTleft{0mm}
\setlength\LTright\fill
\small
\begin{longtable}{|C{15mm}|C{15mm}|C{25mm}|L{95mm}|}
  \caption{Полный пакет Modbus с функцией \\MB\_FUNC\_OTHER\_REPORT\_SLAVEID. Запрос} \label{t:req:MB_FUNC_OTHER_REPORT_SLAVEID} \\
  \hline
  \multicolumn{1}{|p{15mm}|}{\centering Номер \mbox{байта}} &
  \multicolumn{1}{p{15mm}|}{\centering Номера бит} & 
  \multicolumn{1}{p{25mm}|}{\centering Принимаемые значения} &
  \multicolumn{1}{p{95mm}|}{\centering Описание поля} \\\hline
  \endfirsthead
  \caption*{Продолжение таблицы \ref{t:req:MB_FUNC_OTHER_REPORT_SLAVEID}} \\
 \hline
  \multicolumn{1}{|p{15mm}|}{\centering Номер \mbox{байта}} &
  \multicolumn{1}{p{15mm}|}{\centering Номера бит} & 
  \multicolumn{1}{p{25mm}|}{\centering Принимаемые значения} &
  \multicolumn{1}{p{95mm}|}{\centering Описание поля} \\\hline
  \endhead
\rowcolor{ColorModbusSlaveAddr}%
    0  & 7-0 & 0x0A  &  Код манипулятора на шине Modbus\\ \hline  
\rowcolor{ColorModbusFunct}%
    1  & 7-0 & 0x11  &  Чтение информации об устройстве MB\_FUNC\_OTHER\_REPORT\_SLAVEID \\ \hline  
\rowcolor{ColorModbusChecksumm}%    
    2  & 7-0 & 0xC7  &  Контрольная сумма. Биты: [\ \ 7:\ \ 0]. \\ \hline 
\rowcolor{ColorModbusChecksumm}%    
    3  & 7-0 & 0x1C  &  Контрольная сумма. Биты: [15:\ \ 8]. \\ \hline
\end{longtable}
\normalsize



\renewcommand\tablename{\hspace{20mm} Таблица}
\setlength\LTleft\fill
\setlength\LTright{0mm}
\small
\begin{longtable}{|C{15mm}|C{15mm}|C{30mm}|L{80mm}|}
\caption{Полный пакет Modbus с функцией \\MB\_FUNC\_OTHER\_REPORT\_SLAVEID. Ответ} \label{t:answer:MB_FUNC_OTHER_REPORT_SLAVEID} \\
  \hline
  \multicolumn{1}{|p{15mm}|}{\centering Номер байта} &
  \multicolumn{1}{p{15mm}|}{\centering Номер бита} &
  \multicolumn{1}{p{30mm}|}{\centering Принимаемые значения} &  
  \multicolumn{1}{p{80mm}|}{\centering Описание} \\\hline
  \endfirsthead
  \caption*{\hspace{20mm} Продолжение таблицы \ref{t:answer:MB_FUNC_OTHER_REPORT_SLAVEID}} \\
 \hline
  \multicolumn{1}{|p{15mm}|}{\centering Номер байта} &
  \multicolumn{1}{p{15mm}|}{\centering Номер бита} &
  \multicolumn{1}{p{30mm}|}{\centering Принимаемые значения} &  
  \multicolumn{1}{p{80mm}|}{\centering Описание} \\\hline
  \endhead
\rowcolor{ColorModbusSlaveAddr}%
     0  & 7-0 & 0x0A  &  Код манипулятора на шине Modbus\\ \hline  
\rowcolor{ColorModbusFunct}%
     1  & 7-0 & 0x11  &  Чтение информации об устройстве MB\_FUNC\_OTHER\_REPORT\_SLAVEID \\ \hline   
     2  & 7-0 & 0x10  &  Размер пакета (16 байт). Для данного пакета [3...18] (с 3 до 18 байта включительно). Данный байт формирует протокол Modbus~RTU.\\ \hline
\rowcolor{ColorModbusData}%
     3  & 7-0 & 0xFF  &  \\ \hline 
\rowcolor{ColorModbusData}%     
     4  & 7-0 & 0x44  & 'D' \\ \hline 
\rowcolor{ColorModbusData}%     
     5  & 7-0 & 0x65  &  'e' \\ \hline 
\rowcolor{ColorModbusData}%     
     6  & 7-0 & 0x76  &  'v'\\ \hline 
\rowcolor{ColorModbusData}%    
    7  & 7-0 & 0x20  &  '\ \ '\\ \hline
\rowcolor{ColorModbusData}%    
    8  & 7-0 & 0x76  &  'v'\\ \hline
\rowcolor{ColorModbusData}%    
    9  & 7-0 & 0x2E  &  '.'\\ \hline
\rowcolor{ColorModbusData}%    
    10  & 7-0 & 0x31  &  '1'\\ \hline
\rowcolor{ColorModbusData}%    
    11  & 7-0 & 0x2E  &  '.'\\ \hline
\rowcolor{ColorModbusData}%    
    12  & 7-0 & 0x30  &  '0'\\ \hline
\rowcolor{ColorModbusData}%    
    13  & 7-0 & 0x2E  &  '.'\\ \hline
\rowcolor{ColorModbusData}%    
    14  & 7-0 & 0x30  &  '0'\\ \hline
\rowcolor{ColorModbusData}%    
    15  & 7-0 & 0x30  &  '0'\\ \hline
\rowcolor{ColorModbusData}%    
    16  & 7-0 & 0x61  &  'a'\\ \hline 
\rowcolor{ColorModbusData}%    
    17  & 7-0 & 0x0A  &  '\textbackslash n' \\ \hline
\rowcolor{ColorModbusData}%    
    18  & 7-0 & 0x00  & Конец с-строки (0x00)\\ \hline 
\rowcolor{ColorModbusChecksumm}%
    19  & 7-0 & 0x1A  & Контрольная сумма. Биты: [\ \ 7:\ \ 0]. \\ \hline  
\rowcolor{ColorModbusChecksumm}%    
    20  & 7-0 & 0x9A  & Контрольная сумма. Биты: [15:\ \ 8]. \\ \hline  
\end{longtable} \normalsize













\begin{longtable}{|p{10mm}|p{55mm}|p{66mm}|p{10mm}|p{12mm}|}
  \caption{Регистр OCP\_DEBUG\_STATUS} \label{t:bytefield:1} \\
  \hline
  \multicolumn{5}{|p{158mm}|}{%
    \begin{tabular}{p{39.5mm}p{39.5mm}p{39.5mm}p{39.5mm}}
      \textbf{Address Offset} & 0x0000\ FF0C  &  &  \\
      \textbf{Physical Address} & 0x5600\ FF0C  & Instance & SGX \\
      \textbf{Description} & Status of debug.  &  &  \\
      \textbf{Type} & RW  &  &  \\
    \end{tabular}%
  } \\%
  \hline%
  \multicolumn{5}{|p{158mm}|}{%
   \resizebox{174mm}{30mm}{%
  \hspace{-2.2mm}%
  \renewcommand{\arraystretch}{1}\addtolength{\tabcolsep}{-1.215mm}%
  \definecolor{Gray}{gray}{0.8}%
    \begin{tabular}{cccccccc|cccccccc|cccccccc|cccccccc}% 
   \cellcolor{Gray}31 & \cellcolor{Gray}30 & \cellcolor{Gray}29 & \cellcolor{Gray}28 & \cellcolor{Gray}27 & \cellcolor{Gray}26 & \cellcolor{Gray}25 & \cellcolor{Gray}24 & 23 & 22 & 21 & 20 & 19 & 18 & 17 & 16 & \cellcolor{Gray}15 & \cellcolor{Gray}14 & \cellcolor{Gray}13 & \cellcolor{Gray}12 & \cellcolor{Gray}11 & \cellcolor{Gray}10 & \cellcolor{Gray}9 & \cellcolor{Gray}8 & 7 & 6 & 5 & 4 & 3 & 2 & 1 & 0 \\ \hline
   \multicolumn{1}{c}{\rotatebox{90}{CMD\_DEBUG\_STATE}}&
   \multicolumn{1}{|c}{\rotatebox{90}{CMD\_RESP\_DEBUG\_STATE}}&
   \multicolumn{1}{|c}{\rotatebox{90}{TARGET\_IDLE}}&
   \multicolumn{1}{|c}{\rotatebox{90}{RESP\_FIFO\_FULL}}&
     \multicolumn{1}{|c}{\rotatebox{90}{CMD\_FIFO\_FULL}}&
     \multicolumn{1}{|c}{\rotatebox{90}{RESP\_ERROR}}&
     \multicolumn{5}{|c}{\rotatebox{90}{WHICH\_TARGET\_REGISTER\ }}&
     \multicolumn{3}{|c}{\rotatebox{90}{TARGET\_CMD\_OUT}}&
     \multicolumn{1}{|c}{\cellcolor{Gray}\rotatebox{90}{RESERVED}}&
     \multicolumn{1}{|c}{\rotatebox{90}{INIT\_MWAIT}}&
     \multicolumn{2}{|c}{\rotatebox{90}{INIT\_MDISCREQ}}&
     \multicolumn{1}{|c}{\rotatebox{90}{INIT\_MDISCACK}}&
     \multicolumn{1}{|c}{\rotatebox{90}{INIT\_SCONNECT2}}&
     \multicolumn{1}{|c}{\rotatebox{90}{INIT\_SCONNECT1}}&
     \multicolumn{1}{|c}{\rotatebox{90}{INIT\_SCONNECT0}}&
     \multicolumn{2}{|c}{\rotatebox{90}{INIT\_MCONNECT}}&
     \multicolumn{2}{|c}{\rotatebox{90}{TARGET\_SIDLEACK}}&
     \multicolumn{2}{|c}{\rotatebox{90}{TARGET\_SDISCACK}}&
     \multicolumn{1}{|c}{\rotatebox{90}{TARGET\_SIDLEREQ}}&
     \multicolumn{1}{|c}{\rotatebox{90}{TARGET\_SCONNECT}}&
     \multicolumn{2}{|c}{\rotatebox{90}{TARGET\_MCONNECT}} \\ 
    \end{tabular}
}%
 }%
 \\
   \hline%
  \endfirsthead
\caption*{Продолжение таблицы \ref{t:bytefield:1}} \\
  \hline
   \textbf{Bits} & \textbf{Field Name} & \textbf{Description} & \textbf{Type} & \textbf{Reset} \\ \hline
  \endhead
\textbf{Bits} & \textbf{Field Name} & \textbf{Description} & \textbf{Type} & \textbf{Reset} \\ \hline
31 & CMD\_DEBUG\_STATE & \makecell[tl]{Target command state-machine\\0x0: Idle\\0x1: Accept command.} & RW & - \\ \hline
30 & CMD\_RESP\_DEBUG\_STATE & \makecell[tl]{Target response state-machine\\0x0: Send accept.\\0x1: Wait accept.} & RW & - \\ \hline
29 & TARGET\_IDLE            &Target idle & R & - \\ \hline
28 & RESP\_FIFO\_FULL & Target response FIFO full & R & - \\ \hline
27 & CMD\_FIFO\_FULL & Target command FIFO full & R & - \\ \hline
26 & RESP\_ERROR & Respond to OCP with error, which could be caused by either address misalignment or invalid byte enable. & R & - \\ \hline
25:21 & WHICH\_TARGET\_REGISTER & Indicates which OCP target registers to read & RW & \footnotesize{0bxxxxx} \\ \hline
20:18 & TARGET\_CMD\_OUT & \makecell[tl]{Command received from OCP\\Read 0x0: Command WRSYS received\\Read 0x1: Command RDSYS received\\Read 0x2: Command WR\_ERROR\\ received\\Read 0x3: Command RD\_ERROR\\ received\\Read 0x4: Command \\CHK\_WRADDR\_PAGE received.\\Not used.\\Read 0x5: Command \\CHK\_RDADDR\_PAGE received.\\Not used.\\Read 0x6: Command \\TARGET\_REG\_WRITE received.\\Read 0x7: Command \\TARGET\_REG\_READ received} & R & 0bxxx \\ \hline

\end{longtable}




%\newcolumntype{L}{>{$}l<{$}}



%\newcolumntype{L}{>{\verb|}l<{|}} }}



%\newcolumntype{L}{>{  \resizebox{30mm}{\height}{             }l<{    } }}


%\resizebox{30mm}{\height}{ ghfhfghfhl;';l'l;'l;';oioioifhghth }
     
     
     

%\resizebox{30mm}{\height}{  \verb|kghkhjnbvnvbnjhmnhjmnmnmnmnbmnmb| }

%\resizebox{30mm}{\height}{ \usebox{ \verb|kghkhjnbvnvbnjhmnhjmnmnmnmnbmnmb|} }





%\begin{tabular}[t]{lL}
%\resizebox{30mm}{!}{ ghfhfghfhfhghth } & 123, \\
%если & a=b,  \\
%если & a<b, 
%\end{tabular}




%\begin{tabular}{ ll }
%sample &
%\begin{minipage}{3in}
%\begin{verbatim}
%<how>
%%   <to value="make" />
 %  <this value="work" />
%</how>
%\end{verbatim}
%\end{minipage}
%\end{tabular}




%\newcolumntype {I}{ %
%> {\begin{verbatim}\raggedright\arraybackslash}
%l
%<{\end{verbatim}}
%} % 



%\begin{tabular}{ lI }
%sample & gfgdfgdfg
%\end{tabular}




%\begin{tabularx}{100mm}{|l|I|l|}
%\resizebox{30mm}{!}{ ghfhfghfhfhghth } & 123, \\
%если & a=b, & если  \\
%если & hghg &  hgfhgh
%\end{tabularx}




%\String{dfghfdgfd} 11111%




%\newcolumntype{L}{>{\Verb|}l<{|}}


%\ResI{hfghfg;'kj;kl;hgoiioiuouioiuoiuofhgfh}

%\newcolumntype{L}{>{ \begin{itemizeI}  }l<{ \end{itemizeI} }}

%\begin{itemizeI}
%iuiuiui
%\end{itemizeI}


%\newcolumntype{L}{>{ \begin{itemizeI}\raggedright\arraybackslash  }l<{ \end{itemizeI} }}


%\ResII{если}{a=b}{то}{|a-b|=0}


%\begin{tabular}[t]{llll}
%\resizebox{30mm}{\height}{ ghfhfghfhl;';l'l;'l;';oioioifhghth } & 1_2_3, & то & |a-b|=a-b \\
%\ResIII{еnmnbmm_ngghсли\%}{a=b}{то}{a-b=0}
%если & a=b, & то & a-b=0 \\
%если & a<b, & то & a-b=b-a
%\end{tabular}


%\spformedboxmmI{10mm}{c}{bvnbnmnbjghjhbjkhbjkhb}























