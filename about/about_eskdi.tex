\section{Сведения о расширении eskdi  Сведения о расширении eskdi  Сведения о расширении eskdi}
\sectionmark{Сведения о расширении eskdi  Сведения о расширении eskdi  Сведения о расширении eskdi}


\subsection{Назначение расширения Назначение расширения Назначение расширения Назначение расширения}

Расширение \textbf{eskdi} предназначено для оформления текстовой документации согласно ГОСТ~2.105–95 в среде \LaTeX.

Данное расширение не является оригинальным. Уже существуют, по крайней мере, пакеты \textbf{eskdx} и \textbf{eskdpz}.
История создания данного расширения следующая: решил облегчить себе жизнь, избавившись от MS~World.
Сделал обзор в Интернете, скачал \textbf{eskdx} и начал экспериментировать. Затем, при подробном рассмотрении оказалось, что через местный нормоконтроль (бессмысленный и беспощадный) такой документ не пройдёт по следующим причинам:
\begin{itemize}
  \item в содержании после слова ''Приложение А'' не такое выравнивание после переноса на следующую строку названия приложения; 
  \item нумерованные и ненумерованные списки должны начинаться без отступа от края, если произведён перенос на другую строку (как здесь);
  \item не хватало некоторых полей в основных надписях и др.
\end{itemize}

Попытки переделать этот пакет под свои нужды не привели к желаемым результатам: он оказался слишком тяжеловесным и трудным в восприятии. Взять хотя бы то, что все формы там нарисованы средствами \LaTeX, то есть чтобы задать отрезок надо дописать в нужном месте соответствующую команду и координаты концов отрезка (в относительных единицах).

Через некоторое время нашёл в Интернете пакет \textbf{eskdpz}\footnote{Это сноска: Через некоторое время нашёл в Интернете пакет \textbf{eskdpz}}. Как я понял история этого пакета такова: сначала сделали
стилевые файлы в ''LyX'' (это почти визуальный редактор – оболочка для \LaTeX\ бесплатный) затем перенесли эти стилевые файлы на \LaTeX\ и назвали \textbf{eskdpz}\footnote{Это сноска: стилевые файлы на \LaTeX\ и назвали \textbf{eskdpz}}. (для пояснительных записок к дипломным и курсовым). Отличия \textbf{eskdpz} от \textbf{eskdx} в том, что:
\begin{itemize}
  \item \textbf{eskdpz} гораздо проще, то~есть с ним легче разобраться (и функций меньше, но обычно чертежи лучше рисовать не в \LaTeX);
  \item формы и основные надписи выполнены в виде рисунков заднего плана.
\end{itemize}
У \textbf{eskdpz} та же проблема с нумерованными и ненумерованными списками, приложениями и основными надписями. Расширение \textbf{eskdi} сделан на основе \textbf{eskdpz}. Этот документ свёрстан при помощи расширения \textbf{eskdi}.

Вообще, \textbf{eskdi} предназначено больше для инженеров – практиков\footnote{Это сноска: предназначено больше для инженеров – практиков}, которым приходится заниматься разработкой сложных по структуре инструкций в условиях сурового нормоконтроля.




\subsection{Основные особенности пакета eskdi}

Основные особенности \textbf{eskdi}:
\begin{enumerate}
  \item рамки и штампы титульного листа (Ф2.105–1), первого листа (Ф2.106–9), второго листа (Ф2.106–9а), листа регистрации изменений (Ф2.503–3) нарисованы в \LaTeX, причем надписи при смене шрифта не выползают за границы отведённых полей;
  \item можно свободно менять шрифты и стиль начертания в рамках;
  \item в содержании можно печатать названия глав, пунктов, подпунктов, попподпунктов, приложений, подпунктов приложений, подподпунктов приложений причём приложения форматируются так, как требует ГОСТ~2.105–95 в интерпретации местного нормоконтроля (можно поменять, немного поэкспериментировав с стилевыми файлами);
  \item последующие строки в нумерованных и ненумерованных списках выполняются без абзацного отступа (как требует ГОСТ~2.105–95 в интерпретации местного нормоконтроля);
  \item списки в списках поддерживаются;
  \item сноски частично поддерживаются (\LaTeX не приспособлен к постраничной нумерации сносок поэтому иногда могут быть ошибки в нумерации);
  \item названия рисунков и таблиц выполнены с помощью обновлённого пакета \textbf{caption} в замену устаревшего и не рекомендованного для использования \textbf{caption2};
  \item введён режим альтернативной вёрстки (без рамок и штампов но с колонтитулами);
  \item введён режим альтернативной вёрстки для двусторонней печати (без рамок и штампов, в нижнем колонтитуле печатается только номер страницы и децимальный номер);  
  \item работает с дистрибутивом ''\TeX~Live~2014'' (c ''\TeX~Live~2013'' и ниже не работает, с  ''MiKTeX~х.x'' не работает);
  \item работают гиперссылки в содержании;  
  \item кодировка проекта – ''utf8'' (орфографию можно проверять программой ''Texmaker'');
  \item поддерживаются форматы листов A1 (горизонтальный), A2 (горизонтальный), A3 (горизонтальный) : теперь не нужно заменять листы сторонними программами, если требуется вставить формат, отличный от A4 и не бьются гиперссылки при этом;
  \item сборка осуществляется только ''\textbf{xelatex}'', то есть на практике осуществлена полная поддержка работы с шрифтами ''Times New Roman'', ''OpenGost Type B TT'', частично с устаревшим ''GOST type B'';
  \item поддерживается масштабирование документа путём включение в оглавление подпунктов, подподпунктов и т.д;
  \item реализована полная поддержка приложений, то есть корректная нумерация глав, подглав, рисунков, формул в приложениях, корректные ссылки на главы, рисунки, формулы приложений из основной части и из самих приложений;
  \item пакет дополнен широким набором примеров (по типу: найди подходящий и сделай по аналогии), которым является данный документ (основная масса примеров сосредоточена в приложениях);
  \item реализована поддержка ''продвинутой'' двусторонней печати (можно оставлять в документе белые страницы, не включенные в нумерацию, например между титульным листом и листом с основной надписью);
  \item поддерживаются шрифты 12pt и 14pt.
\end{enumerate}


\subsection{Открытые вопросы в использовании \LaTeX}

При разработке пакета и примеров к нему возникли следующие не решенные вопросы:
\begin{enumerate}
  \item как сделать сноски с нумерацией, начинающейся с единицы на каждой странице;
  \item как сделать наклонные русские буквы в формулах и вообще шрифт поменять (при использовании xelatex);
  \item как поменять шрифт в окружении "Verbatim" (при использовании xelatex);
  \item как избавиться от глюков русского языка в окружении "listings" (переставляет русские слова и символы, как в приложении~\ref{app:matlab}).
\end{enumerate}




