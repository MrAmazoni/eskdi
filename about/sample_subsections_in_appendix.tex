\section{Пример включения подразделов из приложений в оглавление}
\sectionmark{Пример включения подразделов из приложений в оглавление}

\subsection{Подраздел 1}

\subsubsection{Подподраздел 1 подраздела 1}

\paragraph{Параграф 1}
\begin{itemize}
  \item в содержании после слова ''Приложение А'' не такое выравнивание после переноса на следующую строку названия приложения; 
    \itemb нумерованные и ненумерованные списки должны начинаться без отступа от края, если произведён перенос на другую строку (как здесь);
    \itemb не хватало некоторых полей в основных надписях и др.
  \item в содержании после слова ''Приложение А'' не такое выравнивание после переноса на следующую строку названия приложения; 
    \itemb нумерованные и ненумерованные списки должны начинаться без отступа от края, если произведён перенос на другую строку (как здесь);
    \itemb не хватало некоторых полей в основных надписях и др.
\end{itemize}

\paragraph{Параграф 2}
\begin{enumerate}
  \item в содержании после слова ''Приложение А'' не такое выравнивание после переноса на следующую строку названия приложения; 
    \itemb нумерованные и ненумерованные списки должны начинаться без отступа от края, если произведён перенос на другую строку (как здесь);
    \itemb не хватало некоторых полей в основных надписях и др.
  \item в содержании после слова ''Приложение А'' не такое выравнивание после переноса на следующую строку названия приложения; 
    \itemb нумерованные и ненумерованные списки должны начинаться без отступа от края, если произведён перенос на другую строку (как здесь);
    \itemb не хватало некоторых полей в основных надписях и др.
\end{enumerate}



\subsection{Подраздел 2}

\subsubsection{Подподраздел 1 подраздела 2}

\paragraph{Параграф 1}

\paragraph{Параграф 2}