\section{Установка пакета eskdi}
\sectionmark{Установка пакета eskdi}



\subsection{ОС ''Windows''}

\subsubsection{Быстрый старт}

Распакуйте архив "eskdi x.xx.7z" в локальную директорию. Русские буквы и пробелы в пути к этой  директории не желательны.

Запустите на выполнение  файл "make.bat". Запустите на выполнение ещё раз. В директории появится файл "about.pdf".



\subsubsection{Интеграция в TeX Live 2014}

Для того, чтобы каждый раз не копировать из проекта в проект все стилевые файлы ("*.sty"), можно поместить их в директорию  "TeX Live", предназначенную именно для такого случая. Отметим, что при таком подходе могут быть трудности со сборкой ранее сделанных документов с помощью ранних версий eskdi. Нужно по крайней мере запоминать какой документ был собран какой версией пакета.

Создайте каталог "eskdi" в директории \verb|"c:\texlive\2014\texmf-dist\tex\latex"| и скопируйте туда стилевые файлы.

Запустите программу "TeX Live Manager 2014" и обновите базу данных согласно рисунку~\ref{p:texlive2014_update_files}.

\begin{figure}[H]\center
  \captionsetup{singlelinecheck=true} %центрируем подрисуночную подпись
  \includegraphics*[scale=0.65]{./about/texlive2014_update_files}
  \caption{Обновление базы файлов в ''TeX Live 2014''} \label{p:texlive2014_update_files}
\end{figure}

Для проверки работы установленных файлов, сотрите в папке с "about.tex" все стилевые файлы, и файл "about.pdf".
Запустите на выполнение  файл "make.bat". Запустите на выполнение ещё раз. В директории появится файл "about.pdf".




\subsection{ОС ''Linux''}


\subsubsection{Быстрый старт}
Не была опробована под Linux.

Распакуйте архив "eskdi x.xx.7z" в локальную директорию. Русские буквы и пробелы в пути к этой  директории не желательны.

Установите права на запуск для файла "make.sh". Запустите на выполнение  файл "make.sh". Запустите на выполнение ещё раз. В директории появится файл "about.pdf".


\subsubsection{Интеграция в TeX Live 2014}
Не была опробована под Linux.














