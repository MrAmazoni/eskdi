\section{Команды eskdi}
\sectionmark{Команды eskdi}


\subsection{Настройки документа eskdi}

\noindent \verb|"MakeStamp"|~– Включение рамок в докуументе.

\noindent \verb|"twoside"|~– Включение двусторонней печати.

\noindent \verb|"SubSectInToc"|~– Включение глав и подглав в содержание.

\noindent \verb|"SubSubSectInToc"|~– Включение глав, подглав, подподглав в содержание.

\noindent \verb|"ParagraphInToc"|~– Включение глав, подглав, подподглав, параграфов в содержание.

\noindent \verb|"MakeEmptyStamp"|~– Печатается вместо рамок только децимальный номер внизу и номер страницы. Только для двусторонней печати.

\noindent \verb|"NumIntoSections"|~–  Включение нумерации формул, рисунков, таблиц внутри глав.



Примеры использования:
\begin{lstlisting}[language=TeX, style=FrameStyle]
\usepackage{eskdi}
\end{lstlisting}
или
\begin{lstlisting}[language=TeX, style=FrameStyle]
\usepackage[MakeStamp]{eskdi}
\end{lstlisting}
или
\begin{lstlisting}[language=TeX, style=FrameStyle]
\usepackage[MakeStamp, SubSectInToc]{eskdi}
\end{lstlisting}
или
\begin{lstlisting}[language=TeX, style=FrameStyle]
\usepackage[MakeStamp, SubSubSectInToc]{eskdi}
\end{lstlisting}
или
\begin{lstlisting}[language=TeX, style=FrameStyle]
\usepackage[twoside, MakeStamp, ParagraphInToc]{eskdi}
\end{lstlisting}
или
\begin{lstlisting}[language=TeX, style=FrameStyle]
\usepackage[SubSectInToc]{eskdi}
\end{lstlisting}
или
\begin{lstlisting}[language=TeX, style=FrameStyle]
\usepackage[twoside, ParagraphInToc]{eskdi}
\end{lstlisting}
или
\begin{lstlisting}[language=TeX, style=FrameStyle]
\usepackage[twoside, MakeEmptyStamp, ParagraphInToc]{eskdi}
\end{lstlisting}


\subsection{Управление шрифтами документа eskdi}


\noindent \verb|"\gostSetRomanfont{}%"|~– Выбор шрифта "romanfont".  Располагать в преамбуле.

\noindent \verb|"\gostSetSansfont{}%"|~– Выбор шрифта "sansfont".  Располагать в преамбуле.

\noindent \verb|"\gostSetMonofont{}%"|~– Выбор шрифта "monofont".  Располагать в преамбуле.

\noindent \verb|"\gostSetMainfont{}%"|~– Выбор шрифта "mainfont".  Располагать в преамбуле.
Пример:
\begin{lstlisting}[language=TeX, style=FrameStyle]
\gostSetRomanfont{Times New Roman}%
\end{lstlisting}




\subsection{Управление рамками документа eskdi}


\noindent \verb|"\gostSetStampLineThick{}"|~– Толщина толстых линий рамки. Располагать в преамбуле. Пример:
\begin{lstlisting}[language=TeX, style=FrameStyle]
\gostSetStampLineThick{0.4mm}%
\end{lstlisting}

\noindent \verb|"\gostSetStampLineThick{}"|~– Толщина тонких линий рамки. Располагать в преамбуле. Пример:
\begin{lstlisting}[language=TeX, style=FrameStyle]
\gostSetStampLineThin{0.2mm}%
\end{lstlisting}


\noindent \verb|"\gostSetStampfont{}"|~– Шрифт текста в рамках. Располагать в преамбуле. Пример:
\begin{lstlisting}[language=TeX, style=FrameStyle]
\gostSetStampfont{Arial}%
\end{lstlisting}





\subsection{Управление подписями документа eskdi}

\noindent \verb|"\gostklgi{}"|~– Децимальный номер.

\noindent \verb|"\gostferstklgi{}"|~– Первичное применение.

\noindent \verb|"\gosttitleobject{}"|~– Название документа. На титульном листе.

\noindent \verb|"\gosttitleobjectI{}"|~– Название документа. В основную надпись.

\noindent \verb|"\gosttitledocument{}"|~– Класс документа. Идёт на титульный лист под название и идёт в основную надпись.

Если документ имеет длинное название, не умещающееся в основную надпись, нужно воспользоваться этим:

\begin{Verbatim}[frame=single]
% Раскомментировать если совсем уж длинное название документа
\renewcommand\SetTitleDocumentInSecondPage{%
  \spboxmm{65}{0}{135}{25}{c}{\parbox{65mm}{\renewcommand
  \baselinestretch{0.6} \centering\footnotesize{Совсем уж 
  длинное, такое длинное, аж до неприличия название для широко 
  известного в узких кругах расширения eskdi (V 1.5b)} 
  \footnotesize{\\ Инструкция по настройке и эксплуатации} }}
}%
\end{Verbatim}


\noindent \verb|"\gostreshenie{}"|~– Идёт в основную надпись в графу для решений.

\noindent \verb|"\gostrazrabotchik{}"|~– Фамилия разработчика. В основную надпись.

\noindent \verb|"\gostrazrabotchikFIO{}"|~– Фамилия и инициалы разработчика. На титульный лист для свободного стиля.

\noindent \verb|"\gostDATErazrabotchik{}"|~– Дата подписи разработчика. В основную надпись.

\noindent \verb|"\gostproveril{}"|~– Фамилия проверяющего. В основную надпись.

\noindent \verb|"\gostDATEproveril{}"|~– Дата подписи проверяющего. В основную надпись.
  
\noindent \verb|"\gostnormokontroler{}"|~– Фамилия нормоконтролёра. В основную надпись.

\noindent \verb|"\gostDATEnormokontroler{}"|~– Дата подписи нормоконтролёра. В основную надпись.

\noindent \verb|"\gostutverdil{}"|~– Фамилия утвердившего. В основную надпись.

\noindent \verb|"\gostDATEutverdil{}"|~– Дата подписи утвердившего. В основную надпись.
  
\noindent \verb|"\gostliteraI{}"|~– Литера. Поле 1.

\noindent \verb|"\gostliteraII{}"|~– Литера. Поле 2.

\noindent \verb|"\gostDATE{}"|~– Дата. На титульный лист для свободного стиля.

\noindent \verb|"\titleLEFT{}"|~– Утверждающая надпись на титульном листе слева. Используется в таком виде:
\begin{Verbatim}[frame=single]
% Раскомментировать если необходима утверждающая надпись на 
%титульном листе
\renewcommand\titleLEFT{%
  \spboxmm{0}{260}{70}{250}{lc}{\parbox{70mm}{\centering
  \normalsize{Утверждён \\} \large{АБВГ.123456.789 --ЛУ} }}
}%
\end{Verbatim}

\noindent \verb|"\titleRIGHT{}"|~– Утверждающая надпись на титульном листе слева. Используется в таком виде:
\begin{Verbatim}[frame=single]
% Раскомментировать если необходима утверждающая надпись на
%титульном листе
\renewcommand\titleRIGHT{%
  \spboxmm{110}{260}{185}{250}{lc}{\parbox{70mm}{\normalsize{
  \begin{center}УТВЕРЖДАЮ\end{center} Генеральный директор\\ 
  ЧОП ''В два глаза''\\ \_\_\_\_\_\_\_\_\_\_\_\_\_\_\_А.Р.Бородач\\
  ''\_\_\_\_\_''\_\_\_\_\_\_\_\_\_\_2015~г.  }  }   }
}%
\end{Verbatim}




\subsection{Управление разделами документа eskdi}


\noindent \verb|"\maketitle"|~– Создание титульного листа.

\noindent \verb|"\makesecondpage"|~– Создание листа с основной надписью.

\noindent \verb|"\regChanges"|~– Создание листа регистрации изменений.

\noindent \verb|"\tableofcontents"|~– Создание оглавления.

\noindent \verb|"\thebibliography{}"|~– Создание списка литературы или библиографии.

\noindent \verb|"\appendix"|~– Начало для приложений.

\noindent \verb|"\section{}"|~– Название главы. Нумерованное.

\noindent \verb|"\section*{}"|~– Название главы. Ненумерованное.

\noindent \verb|"\sectionmark{}"|~– Название подраздела, идущее в колонтитул для свободного стиля.

\noindent \verb|"\subsection{}"|~– Название подглавы. Нумерованное.

\noindent \verb|"\subsubsection{}"|~– Название подподглавы. Нумерованное.

\noindent \verb|"\paragraph{}"|~– Название параграфа. Нумерованное.

\noindent \verb|"\subparagraph{}"|~– Название подпараграфа. Нумерованное.

\noindent \verb|"\subsubparagraph{}"|~– Название подподпараграфа. Нумерованное.




\subsection{Управление форматированием документа eskdi}

\noindent \verb|"\titlePicture{}"|~– Добавление фонового рисунка на титульный лист для свободного стиля. Используется в таком виде:
\begin{Verbatim}[frame=single]
\renewcommand\titlePicture{\AddToShipoutPicture{\AtPageLowerLeft
{\includegraphics[width=210mm, height=297mm, viewport=2.0mm 14.0mm
210.0mm 311.0mm]{\pathtosharedresource background13}}}}%
\end{Verbatim}


\noindent \verb|"\SetEmptyPage"|~– Если текущая страница нечётная, то сделать пустую страницу (не включенную в нумерацию) и сделать следующую страницу нечётной. Если текущая страница нечётная - то команда игнорируется. Режим двусторонней печати. 

\noindent \verb|"\SetEvenPage"|~– Экспериментальная.


%\verb|"\verbatimfont{1}"|~–
%\verb|"\spboxmm"|~–
%\verb|"\spformedboxmm"|~–
%\verb|"\vspboxmm"|~–
%\verb|"\vspformedboxmm"|~–










