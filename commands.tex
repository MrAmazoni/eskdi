% МАКРОСЫ, СПЕЦИФИЧЕСКИЕ ДЛЯ ДАННОГО ДОКУМЕНТА   Подключается в преамбуле

\hyphenation{классицис-ти-чес-кой} % Помогаем LaTeX с переносами
\hyphenation{кирил-ли-чес-ких} % Помогаем LaTeX с переносами





\lstdefinestyle{numbers} {numbers=left, stepnumber=1, numberstyle=\footnotesize} % Настройки нумерации строк для листинга

\lstdefinestyle{MatlabStyle}{% Профиль листинга для Matlab
language=Matlab,%
inputencoding=utf8x,% Глючат русские буквы, но xelatex на другие кодировки ругается
extendedchars=\true,%
%escapeinside={\%*}{*)},
style=numbers,%
showstringspaces=true,%
showspaces=true,% Раскомментировать, если нужна визуализация пробелов
frame={tb},%
lineskip=-1pt,%
basicstyle=\footnotesize,%
commentstyle=\footnotesize,%
xleftmargin=2em,%
keepspaces = true,
breaklines%,%
%escapechar=|
}
\lstdefinestyle{BashStyle}{% Профиль листинга для Bash
language=bash,%
style=numbers,%
showstringspaces=false,%
showspaces=true,% Раскомментировать, если нужна визуализация пробелов
frame={tb},%
lineskip=-1pt,%
extendedchars=\true,%
basicstyle=\footnotesize,%
commentstyle=\footnotesize,%
xleftmargin=2em,%
keepspaces = true,%
basewidth={1pt, 1pt},%
breaklines%
}

\lstdefinestyle{BashStyleII}{% Профиль листинга для Bash
language=bash,%
%style=numbers,%
showstringspaces=false,%
showspaces=true,% Раскомментировать, если нужна визуализация пробелов
frame=tlbr,%
basicstyle=\gostListingfont,%\footnotesize,%
lineskip=-0.5pt,%
extendedchars=\true,%
%commentstyle=\footnotesize,%
xleftmargin=2em,%
xrightmargin=2em,%
keepspaces = true,%
breaklines,%
basewidth={6pt, 6pt}%,
}

\lstdefinestyle{FrameStyleII}{%
        frame=tlbr,%
        showstringspaces=false,%
        basicstyle=\gostListingfont,%
        lineskip=-0.5pt,%
        %escapechar=|,%
        extendedchars=true,%
        %commentstyle=\itshape,%
        commentstyle=,%
        %stringstyle=\bfseries%
        breaklines=true, % перенос слов кода на другую строку
        basewidth={6pt, 6pt},
        xleftmargin=1em,%
        xrightmargin=1em%,%
}%




%showstringspaces=\true,%


\lstdefinestyle{LaTeXStyle}{% Профиль листинга для LaTeX
language=[LaTeX]TeX,%
style=numbers,%
showstringspaces=true,%
%showspaces=true,% Раскомментировать, если нужна визуализация пробелов
frame={tb},%
lineskip=-1pt,%
%extendedchars=true,%
basicstyle=\footnotesize,%
commentstyle=\footnotesize,%
xleftmargin=2em,%
inputencoding=utf8x,
extendedchars=\true,
keepspaces = true,%
breaklines%
}%


\lstdefinestyle{FrameStyle}{%
        frame=tlbr,%
        %escapechar=|,%
        extendedchars=true,%
        %commentstyle=\itshape,%
        commentstyle=,%
        %stringstyle=\bfseries%
        breaklines=true, % перенос слов кода на другую строку
}%


\lstdefinestyle{FrameStyleI}{%
        frame=tlbr,%
        %escapechar=|,%
        extendedchars=true,%
        %commentstyle=\itshape,%
        commentstyle=,%
        %stringstyle=\bfseries%
	basicstyle=\footnotesize,%
        breaklines=false, % перенос слов кода на другую строку
}%








% запись единицы измерения физ. величины
\newcommand{\edizm}[1]{$#1$}

% значение физ. величины с единицей измерения
\newcommand{\numval}[2]{#1~#2}

% запись диапазона величин по ГОСТ
%\newcommand{\gostrng}[4]{#1~#3\ldots#2~#4}
\newcommand{\gostrng}[4]{от #1~#3 до #2~#4}

% пропорциональные круглые скобки
\newcommand{\brkt}[1]{\left(#1\right)}

% пропорциональные квадратные скобки
\newcommand{\brktsq}[1]{\left[#1\right]}

% запись с.к.о
\newcommand{\sko}[1]{\sigma\brktsq{#1}}

% запись квадрата с.к.о
\newcommand{\skosq}[1]{\sigma^2\brktsq{#1}}

% запись мат. ожидания
\newcommand{\mozh}[1]{M\brktsq{#1}}

% запись дисперсии
\newcommand{\disp}[1]{D\brktsq{#1}}

% мат.ож. погрешности деления
\newcommand{\mozhdel}[3]{-2^{-\brkt{#1+1}} \brktsq{1+\frac{#2}{#3}}}

% дисперсия погрешности деления
\newcommand{\dispdel}[3]{
	\frac{1}{3} 2^{-2\brkt{#1+1}}
		\brktsq{1+2\brkt{\frac{#2}{#3}}^2}
}

% дисперсия погрешности умножения
\newcommand{\dispumn}[1]{
	\frac{1}{3} 2^{-2\brkt{#1+1}}
}

% коэффициент предварительного МП в синхр. детектора
\newcommand{\kmp}{K_1 K_{У1} K_{У2} K_{У3}}

% синус с пропорц. скобками
\newcommand{\mysin}[1]{\sin\brkt{#1}}

% косинус с пропорц. скобками
\newcommand{\mycos}[1]{\cos\brkt{#1}}

% мульт. относительная погрешность
\newcommand{\motn}[1]{\delta_{#1}}

% мульт. относительная случайная погрешность
\newcommand{\rmotn}[1]{{\stackrel{\circ}{\delta}}_{#1}}

% мульт. относительная систематическая погрешность
\newcommand{\smotn}[1]{\delta_{s#1}}

% (1 + мульт. отн.)
\newcommand{\motned}[1]{\brkt{1 + \motn{#1}}}

% аддит. относительная погрешность
\newcommand{\aotn}[1]{\gamma_{#1}}

% аддит. относительная случайная погрешность
\newcommand{\raotn}[1]{{\stackrel{\circ}{\gamma}}_{#1}}

% аддит. относительная систематическая погрешность
\newcommand{\saotn}[1]{\gamma_{s#1}}

% абсолютная погрешность
\newcommand{\absp}[1]{\Delta_{#1}}

% абсолютная случайная погрешность
\newcommand{\rabsp}[1]{{\stackrel{\circ}{\Delta}}_{#1}}

% абсолютная систематическая погрешность
\newcommand{\sabsp}[1]{\Delta_{s#1}}

% производная
\newcommand{\deriv}[2]{\frac{\partial #1} {\partial #2}}

% экспонентциальная запись числа
\newcommand{\nexp}[2]{#1 * 10^{#2}}

% перенос ф-лы на другую строку
\newcommand{\eqnewline}[1]{#1 \nonumber\\ {} #1}




% Команда для того чтобы ТУ не переносилось
\newcommand{\ESKDty}[1]{\linebreak[3] \mbox{#1}}

% Ставит черту над выражением (выражение латинскими буквами) - инверсный сигнал
\newcommand{\ESKDoverline}[1]{
  $\overline{\mathrm{#1}}$}

% Ставит черту над выражением (выражение русскими буквами) - инверсный сигнал
\newcommand{\ESKDoverlineRUS}[1]{
  $\overline{#1}$}


% \slshape - переключает шрифт на наклонный

\newcommand\subsectionImit[1]{%
    \vskip 5mm 
    \textbf{#1}
    \vskip 3mm
}



%\graphicspath{{./about/inkscape/}}
































\newcommand{\String}[1]{%
\Verb|#1|}%



\newcommand{\Res}[1]{\resizebox{30mm}{\height}{#1}}




\newcommand{\ResI}[1]{\resizebox{30mm}{\height}{\Verb|#1|}}





\newenvironment{itemizeI}%
{%
\ResI{
}%
{%
}
}%



\newcommand{\ResII}[4]{\resizebox{20mm}{\height}{\Verb|#1|} & \resizebox{20mm}{\height}{\Verb|#2|} & \resizebox{20mm}{\height}{\Verb|#3|} & \resizebox{20mm}{\height}{\Verb|#4|} \\}








\newcommand{\ResIII}[4]{\spformedboxmmI{20mm}{c}{\Verb|#1|} & \spformedboxmmI{20mm}{c}{\Verb|#2|} & \spformedboxmmI{20mm}{c}{\Verb|#3|} & \spformedboxmmI{20mm}{c}{\Verb|#4|} \\}























